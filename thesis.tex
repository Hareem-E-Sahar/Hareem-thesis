% University of Alberta Example Thesis
% By the Rogue's Gallery, Department of Computing Science
% Last updated July 13, 2017

% Note: You will probably have to edit the uathesis.sty file
%       to comment or uncomment bits and pieces
%       (e.g. co-supervisor, externals, etc)

\documentclass[12pt]{report}          % for default format

%%%%%%%%%%%%%%%%%%%%%%%%%
% Package dependencies  %
%%%%%%%%%%%%%%%%%%%%%%%%%
\usepackage{amssymb,amsmath}
\usepackage{url}
\usepackage[hidelinks]{hyperref} % See https://en.wikibooks.org/wiki/LaTeX/Hyperlinks#Customization
\usepackage{tabulary} % Better text wrapping in tables. See https://en.wikibooks.org/wiki/LaTeX/Tables
\usepackage{rotating} % For the 'sidewaystable' environment. See https://en.wikibooks.org/wiki/LaTeX/Rotations
\usepackage{multirow} % For multirow/multicolumn cells in a table. See https://en.wikibooks.org/wiki/LaTeX/Tables#Columns_spanning_multiple_rows
\usepackage{algorithmic}
\usepackage{algorithm}
\usepackage{graphicx}
\usepackage{enumitem}   % for more control over enumerate
\usepackage{parskip} % Extra space between paragraphs
\usepackage{xmpincl} % Seems to be needed when converting to PDF/A, 
                     % particularly on Macs

\usepackage{uathesis}  % Earlier version says this should be last package 
                     % imported. Never checked if this is still true. 
                     % Having this second last before the next one seems fine.

%%%%%%%%%%%%%%%
% biblatex    %
%%%%%%%%%%%%%%%
\usepackage[backend=biber,style=apa,sortcites=true,sorting=nyt]{biblatex}
\usepackage{csquotes}

\addbibresource{refs.bib}
% Needed to prevent the following error with 'apa6' bibliography generation:
% Undefined control sequence. <argument> \mkbibdateapalongextra{year}{month}{day}\iffieldundef {endyear}
% See http://tex.stackexchange.com/questions/115703/yearmonthday-added-to-bibliography-when-using-apa6-with-biblatex-and-biber-bac#115717
\usepackage[american]{babel}
\DeclareLanguageMapping{american}{american-apa}

%%%%%%%%%%%%%%%%%%%%%%%%%%%%
% Package-related tweaks   %
%%%%%%%%%%%%%%%%%%%%%%%%%%%%

% Rename 'Require' and 'Ensure' in pseudocode to 'Input' and 'Output'
\renewcommand{\algorithmicrequire}{\textbf{Input:}}
\renewcommand{\algorithmicensure}{\textbf{Output:}}

% Compensate for 'parskip', which removes indentation
\setlength{\parindent}{15pt}

%%%%%%%%%%%%%%%%%%%%%%%%%%%%%%%%%%%%%%%%%%%
% Title page and Table of Contents Tweaks %
%%%%%%%%%%%%%%%%%%%%%%%%%%%%%%%%%%%%%%%%%%%

%% Correct title for TOC
\renewcommand{\contentsname}{Table of Contents}

% Fill in the following
\title{Thesis Title} % Title can't use formulae, symbols, superscripts, subscripts, greek letters, etc. all of which should be replaced with word substitutes
\author{Author Name}

\degree{\MSc}
%\degree{\PhD} % uncomment respective degree

\dept{Computing Science}  % Write Computing Science or Civil Engineering.

% If you have a specialization, uncomment the following line and enter it below.
% It must correspond with what it says on Bear Tracks. If like most people you
% don't have one, just leave commented
%\field{Specialization Field}

% Put the year that you submitted your thesis below
\submissionyear{\number2014}


\begin{document}

\preamblepagenumbering % lower case roman numerals for early pages
\titlepage % adds title page. Can be commented out before submission if convenient

% The following environments should be filled in as needed. 
% FGSR requires the abstract first, then a preface.
% This can be followed by a dedication or quote (optional).
% FGSR requirements are not clear if you can have both a dedication and quote
% or if you are restricted to one or the other.
% This is then followed by an optional acknowledgements section, then a 
% mandatory table of contents, and then the thesis itself.

% environment for abstract.
\begin{abstract}
Your abstract here. 
The abstract is not allowed to be more than 700 words and cannot include non-text content.
It must also be double-spaced. 
The rest of the document must be at least one-and-a-half spaced.
\end{abstract}


\doublespacing
% \truedoublespacing     Add inside abstract environment if you want that in abstract
% \singlespacing
% \onehalfspacing


% environment for preface
% If you want the preface to not be double-spaced, uncomment the corresponding 
% setting above.
\begin{preface} 
A preface is now mandatory if the thesis includes work that appeared in a journal article (it doesn't mention conference publications) or if ethics approval were needed for any part of the thesis.
Otherwise it is optional.
See the FGSR requirements for examples of how this can look.
\end{preface}

% Below are the dedication page and the quote page. FGSR requirements are not
% clear on if you can have one of each or just one or the other.

% Dedication page
\begin{dedication}
	\vspace*{1in}
	\begin{center}
	         \emph{To the Count} \\
             \emph{For teaching me everything I need to know about math.}
	\end{center}
\end{dedication}

% Quote page
\begin{quotepage}
 \vspace*{1in}
 \begin{center}
	\emph{I think there is a world market for maybe five computers.}
	\begin{flushright}
		-- Thomas J.\ Watson, IBM Chairman, 1943.
	\end{flushright}
 \end{center}
\end{quotepage}

% environment for acknowledgements.
\begin{acknowledgements} 
Put any acknowledgements here.
The acknowledgements can't be more than 2 pages in length.
\end{acknowledgements}

\singlespacing % Flip to single spacing for table of contents settings
               % This has been accepted in the past and shouldn't be a problem
               % Now the table of contents etc.
               
\tableofcontents
\listoftables  % if you have any
\listoffigures % if you have any
               % minimal support for list of plates and symbols (Optional)
%\begin{listofplates}
%...            % you are responsible for formatting this page.
%\end{listofplates}
%\begin{listofsymbols}
%...            % You are responsible for formatting this page
%\end{listofsymbols}
               
% A glossary of terms is also optional, but no such functionality has been added
               
% The rest of the document has to be at least one-half-spaced.
% Double-spacing is most common, but uncomment whichever you want, or 
% single-spacing if you just want to do that for your personal purposes.
% Long-quoted passages and footnotes can be in single spacing
\doublespacing
% \truedoublespacing     Add inside abstract environment if you want that in abstract
% \singlespacing
% \onehalfspacing



\setforbodyoftext % settings for the body including roman numeral numbering starting at 1

%  ... The bulk of your magnificient thesis  goes here ... 
%  hopefully more than two lines! Use standard Latex sectioning commands
%  like \chapter ect. End with the bibliography
% See FGSR requirements for any additional requirements on the body

\chapter{Introduction}

Here is a test reference \cite{Knuth68:art_of_programming}.
These additional lines have been added just to demonstrate the spacing for the rest of the document.

% Other chapters here

% Renaming the bibliography: See http://tex.stackexchange.com/questions/12597/renaming-the-bibliography-page-using-bibtex
\renewcommand\bibname{References}
\clearpage\addcontentsline{toc}{chapter}{\bibname}
     %add the above line to get "References" in the table of contents.
%
\singlespacing % optional;  Bibliography is better in single spacing
               %            but you may choose different
               %            Don't use \singlespacing if your thesis
               %            is already in single spacing
%
\printbibliography

\doublespacing

%\appendix  %  If you have any appendices
            % Use standard Latex sectioning commands
            % like \chapter etc.

\end{document}
